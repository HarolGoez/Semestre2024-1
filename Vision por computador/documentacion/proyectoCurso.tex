\documentclass{article}
\usepackage{graphicx}
\usepackage[left=3cm,right=3cm,top=3cm,bottom=3cm]{geometry}
\usepackage{hyperref}

\title{\textbf{ Clasificación de 5 diferentes tipos de telas usados en la manufactura de camisetas utilizando visión por computador y RNA }}
\author{
    \vspace{8cm} \\
    \textbf{Harol Adrián Góez } \\
    \textbf{Valentina Osorio} \\
    \vspace{2cm} \\
    \small Politécnico Colombiano Jaime Isaza Cadavid\\
    \small Facultad de ingenierías\\
    \small Medellín - Colombia
    \date{ 22 de febrero del 2024 }
}

\begin{document}

\maketitle
\pagebreak

\section{Introducción}

En las últimas décadas el avance tecnológico ha sido notable y en particular, el campo de la visión por computadora 
ha experimentado un crecimiento significativo. Este progreso ha llevado consigo un aumento en las aplicaciones prácticas
tanto en entornos cotidianos como industriales. A medida que los dispositivos han evolucionado,
se ha fortalecido la capacidad de procesamiento, lo que permite satisfacer las exigencias y 
demandas de aplicaciones de visión por computadora de manera más efectiva.

La inteligencia artificial, en particular el aprendizaje automático, ha emergido como una herramienta fundamental 
para abordar una variedad de problemas complejos, ofreciendo soluciones innovadoras que benefician a la sociedad en general. 
Entre las aplicaciones destacadas de estas tecnologías se encuentra la detección y clasificación de materiales textiles,
un proceso crucial en la industria de la moda y la producción de prendas de vestir.
En este contexto, la identificación precisa del tipo de tela utilizada en la fabricación de prendas, como las camisetas, 
reviste una importancia significativa.

Para llevar a cabo esta tarea, se recurre a la utilización de microscopios, 
los cuales permiten obtener imágenes detalladas de la estructura del tejido. Esta información es fundamental 
para determinar con certeza la composición material de la prenda, lo que a su vez influye en aspectos como la calidad,
durabilidad y confort del producto final.

El análisis de imágenes textiles mediante técnicas de visión por computadora ha demostrado ser una metodología 
eficaz y precisa para la clasificación de telas. Este enfoque combina la capacidad de discernimiento visual 
de los expertos humanos con la potencia computacional y la capacidad de aprendizaje de las máquinas,
ofreciendo resultados consistentes y confiables.

En resumen, la integración de tecnologías de vanguardia en la detección de telas textiles representa un avance 
significativo en la mejora de la eficiencia y precisión en la caracterización de materiales. 
Este enfoque no solo es la base para optimizar los procesos industriales, sino que también puede llegar a contribuir al desarrollo de productos de alta calidad
y sostenibles, promoviendo así el progreso en diversos sectores económicos y sociales.

\section{Resumen}

\begin{figure}[ht]
    \centering
    \includegraphics[scale=0.3]{../imagesLatex/microscopio.jpg} 
    \caption{ microscopio para celular }
    \label{fig:1}
\end{figure}

En el presente trabajo se pretende implementar un sistema de visión por computador el cual lea desde un archivo en formato jpg 
una imagen capturada con la cámara de un celular(usando un microscopio analógico) para poder obtener de varias camisetas una
imagen macro de la textura de la tela y de esta forma poder identificar el tipo de tela de la prenda.  La idea es poder clasificar 
hasta 5 tipos de telas usados en camisetas y también generar un sistema que pueda ser capaz de detectarlos de acuerdo a la imagen objetivo.
Todo este proceso se pretende realizar en el curso de visión por computador con el objetivo de aplicar transformaciones de imágenes y otros
métodos comunes en este campo de la electrónica para poder llegar a los resultados deseados.


\section{Solución planteada}

La parte mas importante de este proyecto es poder generar una buena base de datos, para que esto sea posible,
se necesita tomarle demasiadas fotos a camisetas que por lo general se tienen en el closet, ademas de poder clasificar
las imágenes de acuerdo a lo que el fabricante pone en la contextura de las telas, y tener presente que hay combinaciones
de telas en algunas prendas que son fabricadas con 2 o más tipos de materiales diferentes.

\vspace{1cm}
\begin{figure}[htbp]
    \centering
    \includegraphics[scale=0.3]{../imagesLatex/camiseta01.jpg} 
    \caption{Camisetas de lino.}
    \label{fig:2}
\end{figure}

Una vez se tengan las imágenes catalogadas, se debe de construir el dataset para poder preparar la información para poder entrenar la 
red neuronal, el dataSet debe de lucir mas o menos de la siguiente forma

\vspace{1cm}
\begin{table}[h]
    \centering
    \begin{tabular}{||c c c c||} 
        \hline
        \textbf{Nombre de la imagen} & \textbf{Resolución} & \textbf{Etiqueta} & \textbf{Tipo de tela} \\ [0.9ex] 
        \hline\hline
        camiseta1.jpg & 1024x768 & 0 & Algodón \\ 
        \hline
        camiseta2.jpg & 800x600 & 1 & Lino \\
        \hline
        camiseta3.jpg & 1920x1080 & 2 & Poliéster \\
        \hline
        camiseta4.jpg & 1024x768 & 3 & Micro fibra \\
        \hline
        camiseta5.jpg & 4120*2080 & 4 & Seda \\ [2ex] 
        \hline
    \end{tabular}
    \caption{Base de datos inicial para empezar a construir el proyecto}
    \label{table:1}
\end{table}
\vspace{1cm}

La idea inicial es poder construir una base de datos de alrededor 50 archivos donde al menos se pueda tener 10 imágenes por cada tipo
de tela para darle una buena cantidad de datos para que la red neuronal pueda quedar lo mejor entrenada posible y de esa forma
obtener mejores resultados en la capa de salida de la RNA. Luego, se debe de hacer todo el diseño de la red neuronal, 
teniendo presente que los parámetros de entrada(las imágenes) se deben de estandarizar en un formato pre-establecido (igualar los pixeles por fila de todas las imágenes 
y también igualar los pixeles por columna), ademas de escoger el número de capaz de ocultas de la red neuronal y la función de activación pertinente para
este tipo de aplicación.

Cuando el proyecto ya cuente con la base de datos y el diseño de la RNA solamente haría falta hacer el despliegue final de la solución, todo esto se debe de hacer
un aplicativo con python el cual pueda leer una o varias imágenes guardadas en una carpeta ubicada en el computador, y mostrar en pantalla la imagen inicial
y el tipo de tela que fue detectado

\section{Resultados esperados}

Los resultados esperados de este proyecto se mencionan a continuación:

\begin{itemize}
    \item Exactitud: Detección de al menos 5 tipos de telas diferentes usados en la manufactura de camisetas con un porcentaje de acierto superior al 80 por ciento
    \item Compatibilidad : El programa de detección de telas sea compatible con cualquier tipo de resolución de imagen
    \item Rapidez : El sistema debe de hacer todo el proceso en menos de 1 segundo
\end{itemize}


\begin{thebibliography}{9}
    \bibitem{webpage1}
    apposta.com (2024). \emph{tipos de telas} Recuperado de : \url{ https://www.apposta.com/sp/guias/es/tipos-de-telas-camisas.3sp }
    \bibitem{webpage2}
    iniciativaTextil.com (2024). \emph{El top 5 de telas para camisas y blusas} Recuperado de : \url{ https://iniciativatextil.com/telas-para-camisas-blusas }
    \bibitem{webpage3}
    Hoyos Montes, Yaqueline Aide (2020). \emph{ Detección de defectos en fibras textiles utilizando algoritmos de Deep Learning } Recuperado de : \url{https://bibliotecadigital.udea.edu.co/bitstream/10495/15470/1/HoyosYaqueline_2020_DeteccionDefectosFibras.pdf}
    \bibitem{webpage4}
    bncvision.es (2024). \emph{ El rol de la visión artificial en la industria textil } Recuperado de :
    \url{https://bcnvision.es/blog-vision-artificial/} \\
    \url{el-rol-de-la-vision-artificial-en-la-industria-textil/}
\end{thebibliography}

\end{document}